\subsection{Изграждане на работната среда}

Голямо количество усилия бяха положени за изграждането на работанта среда, поради решението да не се използват готови решения,
както и поради нуждата от работа под Linux, тъй като множеството от решения имат целева система Windows.

\subsubsection{Система за изграждане (Make)}

Работната среда е изградена на база \textit{Make} което ни позволява да изградим собствен инструментариум за целите на проекта.
\textit{Make} използва на цели (Targets) и взаимовръзките, които изграждат целите (Dependencies). За всяка цел се изгражда дърво
на зависимостите, като се проверява,
кои зависимости имат нужда от преправяне. \textit{Make} е платформно независима и може лесно да бъде пренесена на почти всяка система.
Единственото нещо от което се нуждае \textit{Make} е описание статично или динамично строго описание на взаимовръзките и начините да се
премине от зависимостите към целите. Това става, чрез дефиниране на командни скриптове, което налага ограничението на средата да бъде
достъпна чрез команден интерфейс.

\subsubsection{Компилатор}

Подбраният комилатор е \textit{GCC ARM NON-EABI}, тъй катое платформно независим комилатор с целева крайна система ARM.
комилаторът е конфигуриран през командният интерфейс при извикването си от \textit{Make},
като конфигурацията е публикувана \autoref{lst:make_config}.
конфигурацията включва използването на хардуерният модул за обработка на числа с плаваща запетая,
изключване на всички оптимизации, които комилаторът прави за подобряване на скоростта и използването на паметта с цел гаранция на изпълнението.

\subsubsection{Програматор}

Подбран е командният пакет за \textit{ST-LINK}, тъй като е платформно независим и позволява командно управление на \textit{ST-LINK V2} устройството,
кето е част от комплекта \textit{32F429IDISCOVERY}. Този пакет позволява презаписване на вътрешната флаш памент на конторлера,
както и стартиране и управление на порт за дебък, който да бъде използван за дебъгера.


\subsubsection{Дебъгер}

Подбран е дебъгер GDB, тъй като е платформно независим и позволява командно базирано дебъгване през дебъг порта отворен с помощта на \textit{ST-LINK}.
Дебъгера позволява условно и безусловно поставяне на Breakpoints на отделни моменти от изпълнението, като се използва генерираният \textit{*.elf} обект,
който е записан в флаш паметта на процесора.

\subsubsection{Получаване и обработка на данни}

Данните се изпращат от микроконтролера, през USART порта и се получават на локалната машина (Linux) използвайки TTL четец, \textit{picocom}.
Данните се записват в \textit{*.log} файл, който се обработва автоматично през дефинираната поточна линия от процеси, като създава \textit{*.csv} файл,
който ще бъде използван от \textit{MATLAB (for Linux)} за обработка на данните и генериране на графики.


\subsubsection{Линкерен скрипт}

Написан е линкерен скрипт, който дефинира правилното разположение на паментта на
контролера и управялва разположението и подравняването на символите и секциите в паметта на микроконтролера.
Линкерният скрипт поставя всички констани символи в флаш паметта на контролера, което гарантира че те няма да бъдат променяни.
Линкерният скрипт също гарантира правилното разположение на вектора на прекъсванията в паметта.
Линкерният скрипт отстранява всички символи свързани с стандартните библиотеки, 
тъй като стандартните библиотеки преполагат наличието на операционна система, 
която да имплементира подфункциите, по този начин софтуерът, който пишем е зависим единствено и само на източниците, които ние сме добавили.
Крайният резултат е независим статично линкнат обект, който ще бъде поставен в микроконтролера.
