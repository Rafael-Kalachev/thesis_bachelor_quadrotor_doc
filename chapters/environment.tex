\section{Използвана среда и инструменти за разработка на платформата}

\subsection{Среда за разработка на софтуера}

Изградената среда за разработка на софтуера е конфигурируема и поддрържа базата микроконтролери от семейство \textit{STM32M4xxx}.
Като основа е използвана автоматичната система за изгращане \textit{GNU make},която позволява насочена обработка на файловете изграждащи софтуера и
документацията, с цел намаляване времето за обработка. \textit{GNU make} свързва всички елементи от средата и последователно изпълнява само нужните
команди с цел намаляване на използваните ресурси. Интерфейсът е команден, което позволява допълнителни нива на автоматизиране на процесите по разработка.

За компилация е използван свободният комилатор на \textit{GNU} 
\textit{GCC (GNU Comiler Collection) ARM NON-EABI (Embedded-Application Binary Interface)}. Тази разновидност на компилатора е неспецифична към
операционна система, което е нужно, тъй като разработваният софтуер няма да работи под операционна система. Тази разновидност на комилатора също е
неспецифична към производител на процесора.
\textit{GCC ARM NON-EABI} Е колекция от свързани инстументи, за разработка на софтуер за системи с ядро \textit{ARM}.
Състои се от комилатор на езика C, асемблатор, линкер, инстументи за преглед и конверсия между стандартни формати двоични файлове.

За връзка с контролера е изполван командният пакет за \textit{ST-LINK} на \textit{STMicroelectronics}.
Пакетът предоставя команди за връзка с програматорът \textit{ST-LINK V2}, чрез който се програмира микроконтролета.
Пакетът се използва също за управление на потрът за дебъг, през който се осъществява дебъг комуникацията.

Като дебъгер е използван свободният дебъгер на \textit{GNU}
\textit{GDB}. Той може да се използва както за локално така и за отдалечено дебъгване.
Тъй като микроконтролера е отдалечен използваме командният пакет за \textit{ST-LINK} да конфигурираме порт за дебъг,
към който се свързваме чрез \textit{GDB}.

За писането на софтуера е изполван терминалният текстов редактор \textit{VIM} поради удобният си команден интерфейс.
За генериране на всички софтуерни тагове, които \textit{VIM} ще използва за подпомагане на процеса за разработка 
е използван софтуера \textit{ctags}, който обработва релевантните файлове и поддържа опростена база данни за всички индентификатори,
имена и тагове, които \textit{VIM} изпозва спрямо контекста.










