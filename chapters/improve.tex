\section{Предложения за надграждане}

При експериментирането с платформата бе забелязано,
че металните профили не потискат добре вибрациите, което създава силни зашумявания в акселерометъра.
Това не се наблюдава в повечето готови решения от подобен размер поради използването на карбонови профили.

За подобряване на стабилността и качеството на управлението се предлага използване на по-високи честоти за измерванията,
както и използване на контролери за управление с дигитален вход за  скоростта на моторите,
тъй като използването на ШИМ не е надеждно за точно задаване на скоростта.

За момента този труд се фокусира върху управлението на ъгъл на завъртане, като един от основните проблеми за цялостно изграждане на беезпилотна платформа с четири ротора.
Като надграждане на този труд се предлага интегрирането на управление по височина, като може да бъде използвано сензорно сливане на сензор за разстояние, и барометър.
Както и използването на GPS и камера за управление чрез позициониране в пространството.

\section{Заключение}

Като част от този труд бяха решени множество иженерни по отношение на създаването на цялостна работна среда за разработка на
софтуер за вградени системи с микроконтролери от семейство STM32M4xxx.

Бяха изградени и/или модифицирани множество библиотеки за математическа обработка с цел изясняване на математическите процедури,
които изграждат често използваните с цел управление алгоритми.

При създаване на системата бе инициализиран хардуера, и бяха инициализирани всички сензори, като често се оказваше,
че документацията на отделните компоненти е недостатъчна и/или непълна.

При създаването на платката за свързване на компонентите бяха решени множество инженерни проблеми свързани с използването на
различни напрежитлни нива на различните компоненти както и коректното разпределение на пиновете на конролета, тъй-като 
микрочипа е част от платка за оценка на функционалностите бе нужно пулно изчитане и сравнение на документацията на платката и контролера,
както и множество от експериментални проверки на изходите за да се гарантира, че пподкачената допълнителна периферия няма да създава проблеми
при комуникацията и четенето на данни.

Като часто от проекта бяха написани драйвъри както и манипулатори (handlers) на прекъсванията с цел възстановяване при
неблагоприятни състояния и грешки в комуникацията. Това намалява аварийните ситуации в случай на грешки.

За сензорите са синтезирани компенсации на слущенията, като са използвани експериментално снети данни.
Синтезиран е филтър на калман, който е използван за оценка на ъгъла на завъртане.

Изградена е платформа за управление на ъгъл на завъртане. Изведен е теоретичен модел на роторите, както и на платформата.
Идентифицирани са параметрите на модела, след което е синтезирано управление.
