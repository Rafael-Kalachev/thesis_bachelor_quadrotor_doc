\section{Въведение}  

През последните години се наблюдава засилване на интереса към създаване и управление на т.нар. квадрокоптери или четирироторни хеликоптери.
Квадрокоптерът представлява малък до среден по размер безпилотен летателен апарат (UAV), който има четири симетрично разположени ротора, обикновено прикрепени в краищата на рамената на платформата. 

В сравнение с други подобни платформи четирироторният апарат притежава здравина, механична простота, стабилност и относително ниска цена, което го прави обект на внимание както по военни, така и по търговски причини.

По-голямата част от четирироторните хеликоптери са конструирани от компоненти за играчки с дистанционно управление и в резултат на това тези устройства нямат необходимата надеждност и производителност, за да бъдат практични експериментални платформи. 

Поради предимствата пред обикновените летателни апарати, потенциала по отношение употребата в открита среда и поради сложната динамика управлението на четирироторен летателен апарат е фундаметнално труден и интересен проблем.



Този труд се концентрира върху цялостното изграждане на система за управление като предоставя
за пример създаване на \enquote{Безпилотна платформа за летателен апарат с четири ротора}.
Ще се наблегне върху направата на софтуер из основи за безпилотния летателен апарат.
По този начин ще бъде демонстрирано как може да се изгради основа за софтуер за управление на непознат, иновативен микроконтролер, за който не съществуват библиотеки. 

С цел да се подобри разбирането за работата на софтуера на системата за управление, ще се елиминира интеграцията с MATLAB (за управление) и използването на специфични много популярни модули, за които има налични множество библиотеки.
Ще бъде разгледан начин за инициализиране и управление на периферията, както методи за обработка на данните, постъпващи от периферията за сформиране на управляващи въздействия.

Избраната система е многомерна и има състояния, които не могат да бъдат измервани директно.
Този труд ще демонстрира изграждането на наблюдателя на състоянията на системата, както и неговата имплементация като част от алгоритъма за управление. 


Този труд няма да разглежда изграждането на система за управление с помощта на Операционна система за реално време. 
Изграждането на ОС за реално време ще бъде плот на отделен бъдещ труд.
Работата в реално време ще бъде осигурена от софтуера на конторлера, но тя няма да бъде разпределена на отделни \enquote{задачи}, а ще се управлява от регулярните прекъсвания на таймера, съпътствани от проста логика и функции, имплементирани по начин, който ще гарантира изпълнение за определеното време.

Като част от този труд е изградена платформа за управление на ъгъл на завъртане на рамо с два ротора.
Тази задача е идентична с една от подзадачите за управление на четирироторна платформа.
