\subsection{Микроконтролер}

Микроконторлерът използван за проекта е с ядро с архитектура \textit{ARM Cortex-M4},
произведен от \textit{STMicroelectronics}.
Модел \texttt{stm32f429ZIT6U}.

\subsubsection{Характеристики}

В следният списък са поместени основните характеристики
на микроконтролера \cite{stmmcudatasheet}.

\begin{itemize} 
    \item Ядро: \textit{32b Arm Cortex-M4} с FPU
    \item Максимална честота на процесора: 180MHz
    \item Флаш Памет: 2048 Kbytes
    \item SRAM: Системна : 256 ( 112 + 16 + 64 + 64 ) Kbytes
    \item Таймери:
    \begin{itemize} 
        \item General Purpose: 10бр. (\change{Add timers})
        \item Advanced control: 2бр. (\change{Add timers})
        \item Basic: 2бр. (\change{Add timers})
    \end{itemize}     
    \item Цомуникационни интерфейси:
    \begin{itemize}
        \item SPI/I2S : 6/2 (пълен дуплекс) 
        \item I2C: 3
        \item USART/UART: 4/4
    \end{itemize} 
    
    \item GPIO: 114бр.
    \item Интерфейс за програмиране: \textit{ST-LINK}
    \item Опаковка: LQFP144 
\end{itemize} 

\subsubsection{Комплект 32F429IDISCOVERY}

Микроконтролерът е час то платка \textit{32F429IDISCOVERY} 
(комплект за оценка на функционалиностите на \textit{STMicroelectronics}).
комплектът предоставя множество функционалности вградени в платката като
8MHz външен осцилатор, Вграден \textit{ST-LINK} програматор,
LCD Дисплей, 2бр. LED, както и удобна връзка към пиновете на
микроконтролера в THT формат.

Важно нещо да се отбележи, е че в комплекта \textit{32F429IDISCOVERY}
има познат проблем с вграденият \textit{ST-LINK} програматор,
който докато USB не е свързано поддържа процесора в \textit{RESET}
през дебъгера.


